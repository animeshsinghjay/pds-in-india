\documentclass{article}
\usepackage[utf8]{inputenc}

\title{\textbf{The Public Distribution System in India}
\linebreak A Case Study}
\author{Animesh Singh}
\date{May 2018}

\begin{document}

\maketitle

\begin{abstract}
This document pin points the flaws in the current PDS system in India and solves those problems by proposing various design solutions. Section 1 provides an introduction to the problem and motivation to propose solutions. Section 2 discusses the stakeholders of the problem followed by Section 3 which is the Needs Analysis where we have discussed the basic set of properties that needs to be satisfied by a PDS. Section 4 illustrates the current PDS structure and its associated problems. Section 5 proposes design solution reforms for the current PDS system. Section 6 contains References.
\end{abstract}

\section{Introduction}

The Public Distribution System (PDS) is India's food security system, established by the Government of India under Ministry of Consumer Affairs, Food and Public Distribution to distribute subsidized food and non-food items to India's poor. 
The Public Distribution System contribute significantly in the provision of food security. Public Distribution System in the country enables the supply of food grains to the poor at a subsidized price. It also helps to control open - market prices for commodities that are distributed through the system. Government accords great importance to the objective of measuring outcomes of PDS so as to ensure that equal distribution system serves up the purpose for which it was set up.\par
India's Public Distribution System is a broad network. The concept of Public Distribution System in India emerged during 1942 for the first time in revised form as a result of shortage of food grains during World War. Subsequently, Government started intrusion in release of food to the people. Public Distribution System in India is more than half-a century old as rationing was first introduced in 1939 in Bombay by the British Government as a measure to ensure equitable distribution of food grains to the urban consumers as a result of increasing prices. Thus, rationing in crisis period particularly during shortage was the historical antecedent to the national policy of stabilization and management of food grains. After independence in 1947, major aim of Government of India has been to deliver food security to all the inhabitants of India. With this objective, Public distribution system was started by Ministry of Consumer Affairs, Food and Civil Supplies. \par
Public Distribution System is a network whereby accessibility of vital supplies is guaranteed which can be easily accessed by the consumers in every part of the country. This is a transaction system where food grain, sugar, and other needed items such as kerosene oil and edible oil are made available to the people of the state at fair price to meet their minimum needs. Regular and timely availability of supplies is assured through close monitoring system to make Public Distribution System an effective instrument against various forces in the open market and to keep under check the inflator tendencies. The main commodities are as follows:
\begin{itemize}
\item Wheat
\item Rice
\item Sugar
\item Kerosene
\end{itemize}
Certain supply on fixed and affordable prices also keeps in control the variable trends of market due to vagaries of whether and subsequent changing prospects of crops. Public Distribution System serves as a steady stable check on market forces and work as an effective soothing factor. \par
In order to efficiently manage and distribute food grains, the government of India has created Ministry of Consumer Affairs, Food and Public Distribution. The ministry has been divided into two departments specifically Department of Food and Public Distribution and Department of Consumer Affairs. The department of food and public distribution is further divided into two parts for the purchase and storage of food grain. When appraising Public Distribution System in India, it has been found that among all states, Tamil Nadu has done marvellous job in implementing the PDS as universal system for the cause of eliminating poverty and improving standard of living of the people living below the poverty line. Timely supply of essential commodities is the basic element for the success of the Public Distribution System. Infrastructure i.e., Fair Price Shops (FPS), godown facilities and employees are other requisites of the PDS. \par
In 1997, the government launched the Targeted Public Distribution System (TPDS) to resolve the problems of poor communities. The prime aim of TPDS is to provide subsidised food and fuel to the poor through a network of ration shops. Food grains such as rice and wheat that are provided under TPDS are procured from farmers, allocated to states and delivered to the ration shop where the beneficiary buys his entitlement. The centre and states share the responsibilities to identify the poor, procuring grains and delivering food grains to beneficiaries. In September 2013, Parliament passed the National Food Security Act, 2013. The Act relies largely on the existing TPDS to deliver food grains as legal entitlements to poor households. This made a shift by making the right to food a justiciable right. In order to understand the implications of this Act, the note maps the food supply chain from the farmer to the beneficiary, identifies challenges to implementation of TPDS, and debated alternatives to improve TPDS.


\section{Stakeholders}

I identify the following stakeholders for the given design problem.

\subsection{Primary Stakeholders}

\begin{enumerate}  
\item Citizens - The section of Indian population which uses and is dependent upon the Public Distribution System for its ration needs are the most vital stakeholder in this design problem. They directly affected by its problems and would be directly benefited by proposed reforms. This section now include just the Below Poverty Line (BPL) members as opposed to the earlier system which was inclusive of APL and BPL members. The line classifying them is also not very well defined as would be discussed later.
\item FPS Employee - The salesperson on the Fair Price Shop is also a direct stakeholder in the system. They are directly affected by the sales of the FPS and the system relies on their integrity. If they try to reap benefits out of the loopholes in the system, they directly be affected by reform measures.
\item Transporter/ Distributor/ Middle Men - They are an important stakeholder as they are the ones which try to reap benefits out of the loopholes in the system and would directly be affected by reform measures.
\end{enumerate}

\subsection{Secondary Stakeholders}

\begin{enumerate}  
\item Government of India - The Government of India is responsible for the correct and just functioning of the PDS system. This responsibility induces a secondary stakeholder.
\item Taxpayers - The taxpayers' money is used to manage the entire PDS system. Hence, they are secondary stakeholders in the process and are indirectly affected by its flaws.
\end{enumerate}


\section{Needs Analysis}

I identify the following set of basic requirements for the given design problem.

\subsection{Price}

The grains should be provided at prices fixed by the government. Every individual should be able to verify the updated prices.
    
 \subsection{Quantity}
 The quantity of grains to be given to each individual should be determined by the government. The FPS shopkeeper should not be able to manipulate that quantity. Also, every individual should be able to verify whether the shopkeeper has given him his rightful share of grain or not.
 
 \subsection{Quality}
 The quality of grains to be given to each individual should be determined by the government in each season. The FPS shopkeeper should not be able to substitute high quality grains supplied by the govt with low quality grains so that he can profit from selling high quality grains in the open market. Every individual should be able to verify whether the shopkeeper has provided him with grains of the quality intended by the government or not.
 
 \subsection{Flexibility}
 The customer should have the option of sending someone on his behalf to procure the grains in case he is physically disabled or even for the sake of convenience. Hence the entire ration for a family can be procured by a single family member; all members need not go the the FPS shop.
 
 \subsection{Portability}
The customer should have the option of changing the ration shop from where  he is supposed to procure the grains.
    
\section{Current PDS and Associated Problems}

The central and state governments share responsibilities to provide food grains to the identified recipients. The centre procures food grains from farmers at a minimum support price (MSP) and sells it to states at central issue prices. It is responsible for transporting the grains to godowns in each state. States hold the responsibility of transporting food grains from these godowns to each fair price shop (ration shop), where the beneficiary buys the food grains at the lower central issue price. Many states further subsidise the price of food grains before selling it to recipients. The Food Corporation of India (FCI) is the nodal agency at the centre that is responsible for transporting food grains to the state godowns.\par
State-level ministries of food and civil supplies control networks of ration shops within their authorities, and are responsible to allocate licenses to the private traders who operate the shops. State governments also issue ‘ration cards’ to their residents (at one time on a nominally universal basis, but more recently on a ‘targeted’ basis), and determine the quantities to which consumers are entitled. These differ from one commodity to the next. The prices are determined by state governments. Under Public Distribution System scheme, each family below the poverty line is eligible for 35 kg of rice or wheat every month, while a household above the poverty line is entitled to 15 kg of food grain on a monthly basis. The Central Government take responsibility for procurement, storage, transportation, and bulk allocation of food grains. State Governments hold the responsibility for distributing the same to the consumers through the established network of Fair Price Shops (FPSs). State governments are also responsible for operational responsibilities such as Allocation and Identification of families below poverty line, Issue of ration cards, Supervision and Monitoring the functioning of Fair Price Shops.\par
In the current system, food grains are transfered from FCI store to states and then to regional levels. Most states rely on registers to keep track of grain given out. This leads to the problems of bogus cards and dummy people without any authentication. There are different ways in which different states of India are implementing the PDS, especially the last mile authentication of sale. All of them involve authentication of the user before giving out the grain. 
\par I identify the various problems that the current PDS design suffers from as follows.

\begin{enumerate}
  \item The FPS shopkeepers earn their income from the commission from purchases and transactions made at the ration shop. When the system of PDS was introduced, it was meant for BPL as well as APL sections of the society. But due to limited resources, the focus shifted to the BPL citizens because of which the number of transactions taking place at the ration shops reduced. This resulted in less profit for the shopkeeper of the FPS. Low payment can motivate shop keepers to go for corrupt methods to increase their income.
  \item Most of the food supply diverges in its transportation between FCI and FPSs. The truck drivers and transporters who get the grains to the ration shops, often keep major chunks of the grains for themselves.
  \item There is lack of awareness and education to people about their rights. Most of the citizens living in rural areas do not even know about how much wheat and rice they are entitled to. Due to this many shopkeepers fool them and keep the rest of the material for themselves.
  \item There can be bribery at the level of consumers itself. It is possible that a consumer pays bribe to the shopkeeper and asks them to provide extra amount (which would ultimately come from someone else's quota).
  \item Middlemen sometimes replace the food grains (which the FCI supplies) with inferior quality grains for profit in reselling the better grains.
  \item Shopkeepers can make bogus cards for ghost people and under their name, sell food grains in open (here black) market.
  \item There is inaccurate identification of beneficiaries. Identifying households who have been granted PDS services is irregular in various states. There is no set criteria as to which a family is categorized as BPL and APL. This ambiguity gives massive scope for corruption and fallouts in PDS systems because those who are actually meant to be benefited are not able to taste the fruits of PDS.
  \item If a state uses Aadhaar based biometric authentication, people who are not able to get their ration themselves usually ask their relatives to get it for them are at a disadvantage because there is no way to provide flexibility in the current system.
\end{enumerate}


\section{Design Solutions for Reforms}

I propose various reforms that can be used to integrate into the current PDS system to solve its flaws. 

\subsection{Adhaar Linked and Digitized Ration Cards (Smart Cards)}

There is immense role of Aadhaar card in public distribution system. It has been realized that one major problem in the implementation of the targeted public distribution system is the inclusion and barring errors in the identification of recipients. It is proposed to integrate the Unique Identification or Aadhaar number with the PDS. The Aadhaar number would be used to precisely identify and authenticate beneficiaries entitled to receive subsidies under the targeted public distribution system and other government schemes. It would help remove duplicate and fake beneficiaries, and make identification of beneficiaries more accurate. This allows online entry and verification of beneficiary data. It also enables online tracking of monthly entitlements and off-take of foodgrains by beneficiaries. Fake cards and duplication of names of members could be averted and distribution of PDS commodities to the deserving will be ensured besides bogus billing. 

\subsection{Computerized Fair Price Shops and Commodity Bags}
FPS can be automated by installing ‘Point of Sale’ device to swap the ration card. It authenticates the beneficiaries and records the quantity of subsidized grains given to a family.
\par There is a need to tackle the problem of leakage in the current system. Therefore, in our system we could use a sealed a bag which would contain fixed amount of grain in it and it would be tracked using a barcode. Hence, leakages due to the middlemen would be reduced keeping the quantity and quality of the grains intact. Since we can track the grains, this would make the person involved accountable. This would also satisfy the features of Quantity and Quality Restrictions as discussed in the Needs Analysis.
\par After verifying the person using smart card, we scan the barcode from the sealed bags hence marking that a particular bag was distributed to a particular person. This would increase transparency and further ensure that no malpractices take place as it increases verifiability. 
\par Also, if anyone is not able to avail the facility of PDS themselve, they could also nominate a fixed no. of persons who could take the supply on their behalf. This satisfies the Portability issue in the current PDS.

\subsection{Direct Benefit Transfer (DBT)}
Under the Direct Benefit Transfer scheme, cash is transferred to the beneficiaries’ account in lieu of foodgrains subsidy component. They will be free to buy food grains from anywhere in the market. For taking up this model, pre-requisites would be to complete digitization of beneficiary data and seed Aadhaar and bank account details of beneficiaries.

\subsection{Use of GPS Technology}
Use of Global Positioning System (GPS) technology to track the movement of trucks carrying foodgrains from state depots to FPS which can help to prevent diversion.

\subsection{SMS Based Monitoring}
This system allows monitoring by citizens so they can register their mobile numbers and send/receive SMS alerts during dispatch and arrival of TPDS commodities. 

\subsection{Web Based Citizens Portal}
Public Grievance Redressal Machineries, such as a toll-free number for call centers to register complaints or suggestions. This facility could be used by the citizens (APL and BPL) and also by the shopkeepers and employees of the PDS system. This would ensure a better and more interactive system of checks and balances.

\subsection{Community Grain Fund (Local Procurement and Distribution)}
One reason for the inefficiencies and corruption that plague the current system is the long distribution chain. Food supplies go waste or are diverted at each step of the process. It also takes a long time for food produce to reach ends of the chain (this also increases the cost of managing the public distribution system). If the local administration takes charge of procurement (unless certain foods are not available locally), it can collect food produce locally. This can then be distributed locally to eligible households. By involving local people in the process, procurement and storage of grain is done at the local level, and could be managed by local women. These locals also identify eligible households through a participatory wealth ranking process. This addresses issues related to exclusion and inclusion, a major problem in the current public distribution system. The fund can also lend money to farmers to cultivate crops and take a part of the produce as repayment in kind, further simplifying the procurement process.

\subsection{Inclusion of the FPS Employees under Government Pay Commission}
In the current system, the shopkeepers get paid by the transactions of the consumers at the ration shops. If they are made government employees who get a fixed amount of salary, the problem of them being underpaid can be solved. Hence, there would be no involvement of retailers; the task would be done by government employees whose work would be periodically checked.
	

\section{References}
\begin{enumerate}  
\item Reetika Khera: "Smarter than Aadhaar: Govt's insistence on
disruptive option is bewildering", Business Standard, 14 March 2018
\item Reetika Khera: "Revival of the Public Distribution System:
Evidence and Explanations"
\item Avishek Pradhan: "PUBLIC DISTRIBUTION SYSTEM IN INDIA: A Success Story?", Siddhartha Law College, Dehradun
\item https://www.civilserviceindia.com/subject/General-Studies/notes/public-distribution-system-functioning-limitations-revamping/
\item Dr.R.Velmurugan, Mrs.D.Lavanya: "Problems in Public Distribution System", Journal of Progressive Research in Social Sciences (JPRSS), 30 April 2015
\end{enumerate}


\end{document}
